\documentclass[11pt,twocolumn]{article}
    % General document formatting
    \usepackage[margin=0.7in]{geometry}
    \usepackage[parfill]{parskip}
    \usepackage[utf8]{inputenc}

    % Related to math
    \usepackage{amsmath,amssymb,amsfonts,amsthm}

\newcommand{\matr}[1]{\mathbf{#1}}

\title{On the Chaining of Helmert transformations}
\author{Kristian Evers \& Lars Stenseng}
\date{2022-08-10}

\begin{document}
\maketitle

\section{Introduction}

The purposes of this document is to demonstrate that a chain of consecutive
Helmert transformations can be combined into one set of parameters with a bit of
algebra.

\section{About the Helmert transformation}

The Helmert transformation formulas as we know it:

\begin{equation}
    \begin{bmatrix}
        x
        \\ 
        y
        \\ 
        z
    \end{bmatrix}^B
%
    =
%
    \begin{bmatrix}
        t_x \\ t_y \\ t_z
    \end{bmatrix}
%
    + (1+s\cdot10^{-6})
%
    \begin{bmatrix}
        1       & -r_{z}  &  r_{y}   \\ 
        r_{z}   &  1      & -r_{x} \\ 
        -r_{y}   &  r_{x}  &  1
    \end{bmatrix}
%
    \begin{bmatrix}
        x \\ y \\ z
    \end{bmatrix}^A
\end{equation}

can also be expressed as

\begin{equation}
    \label{eq:simplehelmert}
    P_{B} = T + c\matr{R}P_{A}
\end{equation}

where $P_{A}$ and $P_{B}$ are the input and output coordinat vectors,
$T$ is the translation vector, $R$ is the rotation matrix and the scale
factor $c$ is expressed as

\begin{equation}
    c = 1+s\cdot10^{-6}
\end{equation}

\section{Chaing transformations}

It is quite common to perform several Helmert transformations
in a row, e.g.

\begin{equation}
    \label{eq:helmert1}
    P_B = T_1 + c_1\matr{R_1}P_A
\end{equation}

\begin{equation}
    \label{eq:helmert2}
    P_C = T + c_2\matr{R}_2P_B
\end{equation}

where the coordinate $P_A$ transformed first with the parameters of the first
step ($T_1$, $c_1$ \& $\matr{R_1}$) and then the parameters of the second step
($T_2$, $c_2$ \& $\matr{R_2}$).

This is computationally heavy and can be avoided by
substituting eq. \ref{eq:helmert1} into eq. \ref{eq:helmert2}:

\begin{equation}
    P_C = T_2 + c_2\matr{R_2} \left( T_1 + c_1 \matr{R_1} P_A \right)
\end{equation}

\begin{equation}
    \label{eq:helmert_substituted}
    P_C = T_2 + c_2\matr{R_2}T_1 + c_2c_1 \matr{R_2}\matr{R_1} P_A
\end{equation}

which can be simplified using the following expressions

\begin{equation}
    \label{eq:t3}
    T_3 = T_2 + c_2\matr{R_2}T_1
\end{equation}

\begin{equation}
    \label{eq:c3}
    c_3 = c_2 \cdot c_1
\end{equation}

\begin{equation}
    \label{eq:r3}
    R_3 = \matr{R_2}\matr{R_1}
\end{equation}

Substituting equations \ref{eq:t3}, \ref{eq:c3} and \ref{eq:r3} into
eq. \ref{eq:helmert_substituted} we get

\begin{equation}
    P_C = T_3 + c_3 \matr{R_3} P_A
\end{equation}

which is on the same form as the Helmert transformation show in
\ref{eq:simplehelmert}.

A chained Helmert transformation can be sped up significantly by first
determining a combined set of transformation parameters using equations
\ref{eq:t3}, \ref{eq:c3} and \ref{eq:r3} and then using them in eq.
\ref{eq:simplehelmert}.

\section{Including velocity}
Following the notation above we introduce the input, $V_A$ and output $V_B$ velocity vectors and the translation rate vector $\dot{T}$, rotation rate matrix $\dot{\matr{R}}$, and the scale rate factor $\dot{c}$.


\begin{equation}
	V_B = V_A + \dot{T} + \dot{c} \dot{\matr{R}} P_A
\end{equation}


\end{document}
